\section{Conclusion}
\label{sec:conc}

In this article, we demonstrate that SPNMF is a very promising model for harmonic/percussive decomposition. Indeed, the SPNMF outperforms three other state of the methods on the medley-dB database~\cite{bittner2014medleydb}. Carrying out an evaluation on a large database allowed us to compare more accurately the performance of the four methods on a large variety of music signals. 

On a large database none of the state of the art methods produces a satisfying harmonic/percussive source separation. The large variety of audio signal makes the task extremely difficult. That said, on a smaller section of the database, the SPNMF showed some promising results (see~\ref{sec:subdata}).

We can say that the information from the drum dictionary built from the database ENST-Drums~\cite{gillet2006enst} is not sufficient to perform a harmonic/percussive source separation on a large scale. Depending of the style of music, some drums share similarities. A possible improvement would be to build genre specific drum dictionaries. In this way, the computation time would be reasonable as the amount information could be reduced and the templates of the dictionary could be a lot more focused on specific type of drums. 


