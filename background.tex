\section{Contexte of harmonic/percussive decomposition}
\label{sec:Background}


Harmonic instruments can be classified into several categories such as the sustained instruments (flute, violin), the hammered instruments (piano, glockenspiel) and the string instruments (guitar, mandolin)~\cite{peeters2003automatic,peeters2003hierarchical}. These instruments can be modeled as a sum of three parts: the Sinusoidal part, the Transients and the Residual, so called  Sines + Transients + Noise (STN) models~\cite{daudet2006review}. The transient of the harmonic instruments show some percussive properties (i.e., fast attack and fast decay). Taking into account the transient part of harmonic instruments is a challenging task which resolution is out of the scope of this article. Here, as in other article of the literature, we rely on the hypothesis that most of the energy of the harmonic signal is in the tonal part. 

Harmonic/percussive source separation is based on the observation that harmonic instruments are tonal sounds very localized in frequency that can be modeled by sum of sinusoidal waves and percussive instruments are wide band transient signals. It has numerous application as a preprocessing step. For example most multi-pitch estimation models~\cite{klapuri2008multipitch}, instruments recognition and melody extraction~\cite{salamon2012melody} algorithms are much more efficient if the influence of the percussive sources is diminished. These algorithms often rely on the analysis of the harmonic structures that are blurred by the percussive instruments. Similarly, beat tracking~\cite{ellis2007beat} and drum transcription algorithms~\cite{paulus2005drum} are more accurate if the harmonic instruments are not part of the signal. Finally, using the Harmonic/Percussive Source Separation (HPSS) algorithm~\cite{fitzgerald2010harmonic} as a preprocessing step increases the performance for singing pitch extraction and voice separation~\cite{hsu2012tandem}.

A well known unsupervised method to extract the harmonic/percussive components consists in applying a median filtering on the spectrogram of the audio signal~\cite{fitzgerald2010harmonic,ono2008separation}. The filtering is made along the temporal atoms to diminish the transient sounds in order to extract the harmonic components. Mutually, the filtering is made along the frequency to reduce the harmonic tonal sounds and to enhance the percussive instruments. The assumption is that the harmonics are considered to be outliers in a temporal frame that contains a mixture of percussive and pitched instruments, similarly, the percussive onsets are considered to be outliers in a frequency frame. This method is often use in the community as it does not require any parameter tuning and is computationally effective. However, this simple approach does not give the best separation results~\cite{canadas2014percussive}.

Another unsupervised method uses a NMF decomposition with specific constraints to distinguish the harmonic part from the percussive components~\cite{canadas2014percussive}. A simple approximation is that percussive instruments are transient sounds with regular spectra (wide band signals) whereas harmonic instruments are tonal sounds with harmonic sparse spectra. From this observation, a frequency regularity and a temporal sparsity constraints are applied during the optimization process to extract the percussive instruments and vice-versa a temporal regularity and a frequency sparsity constraints are applied to extract the harmonic instruments. This method obtains good results against other state of the art methods~\cite{canadas2014percussive} but the hyper-parameters tuning is a tedious process and results depend heavily on the training database.

Finally in~\cite{kim2011nonnegative}, drum source separation is done using a non-negative partial co-factorization. The spectrogram of the signal and the drum-only data (obtain from prior learning) are simultaneously decomposed in order to determine common basis vectors that capture the spectral and temporal characteristics of drum sources. The shared dictionary matrix retrieves the drum signal, however it must be chosen carefully in order to obtain satisfying results. The percussive part of the decomposition is constrained while the harmonic part is completely unconstrained. As a result, it tends to decompose a lot of information from the signal and the decomposition is not satisfying {\MK il faut préciser ici "lot of information and the decomposition is not satisfying"}. In the NMPCF {\MK NMPCF n'est pas defini. On se doute que c'est la methode~\cite{kim2011nonnegative}, mais il faut que ca apparaisse plus clairement}, the dictionary are not fixed {\MK il y a un ou plusieurs dictionnaires dans la NMPCF ?}, we will this method compare it to the SPNMF were the dictionary is fixed {\MK probleme de redaction ici}. 
