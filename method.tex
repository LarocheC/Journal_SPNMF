\section{Structured projective NMF (SPNMF)}
\label{sec:SPNMF}

In this section we present our semi-supervised algorithm for harmonic/percussive source separation.

\subsection{Presentation of the orthogonal and projective NMF}\label{subsec:PNMF}


Using a squared Euclidean distance between the data matrix $V$ and its approximation $WH$, the NMF problem reads:
$$
\min_{W,H\geq 0} \|V - WH\|^2\ , 
$$
where $\|.\|^{2}$ is the squared Euclidean distance.

The aim of the PNMF is to find a non negative projection matrix $P \in \mathbb{R}_{+}^{n \times n}$ such that $V \approx \tilde{V} = PV$. In~\cite{yuanOja2005} Yuan \& al. propose to seek $P$ as an approximative projection matrix under the form $P = WW^{T}$ with $W \in \mathbb{R}_{+}^{n \times k}$ with $ k \leqslant n $. The PNMF problem reads : 
\begin{equation}\label{EqPnmf}
\min_{W \geqslant 0} ||V - WW^{T}V||^2 
\end{equation}

PNMF is similar to the NMF problem and can be simply obtained by replacing the activation matrix $H$ by $W^TV$. It is shown in~\cite{YangOja10} that the PNMF gives a much sparser decomposition than the NMF.

Another very similar approach is the ONMF~\cite{choi}. It consists in solving the following problem: 
\begin{align}
\min_{W \geqslant 0, H \geqslant0} ||V - WH||^2 \quad   \text{s.t}.\quad W^{T}W=I_{k} 
\end{align}%
In this method, orthogonality between nonnegative basis functions is enforced during the optimization process. In theory, it seems that the PNMF and the ONMF lead to similar decompositions, as the $W$ matrix estimated by the PNMF is almost orthogonal (i.e., $\|W^{T}W-I_{k}\|^{2}$ is small). However in practice, enforcing the orthogonality between the base at every iteration is a constraint too strong to decompose audio signal~\cite{laroche2015structured}. 

The sparsity of the dictionary matrix is an interesting property for the decomposition of audio signals and especially for the decomposition of harmonic instruments with very localized harmonic spectra. Contrary to the NMF, the sparsity of the PNMF is an inherent features of the decomposition and contrary to a regularization term, there is no hyper parameter tuning. These key properties of the PNMF motivated us to decompose the harmonic instruments using orthogonal basis functions.


\subsection{Principle of the SPNMF}

The orthogonal basis functions of the PNMF are not flexible enough to decompose a complex audio signal. As stated in~\cite{canadas2014percussive}, harmonic instruments have sparse basis functions whereas percussive instruments have much flatter spectra. As the columns of $W$ are orthogonal, when two sources overlap in the Time-Frequency (TF) plane only one basis function will represent the mixture which is not adequate for efficient separation. To overcome this problem, we propose to add a standard NMF decomposition term to the PNMF. We can expect that most of the harmonic components will be represented by the orthogonal part while the percussive ones will be the regular NMF components. Using a similar model to preliminary work~\cite{laroche2015structured}, let $V$ be the magnitude spectrogram of the input data. The model is then given by
\begin{equation} \label{Cfunction}
V \approx \tilde{V}= V_H + V_{P},
\end{equation}
with $V_P$ the spectrogram of the percussive part and $V_H$ the spectrogram of the harmonic part. $V_H$ is approximated by the PNMF decomposition while $W_P$ is decomposed by some NMF components as :
\begin{equation}
V \approx \tilde{V}= W_{H}W_{H}^{T}V + W_{P} H_{P}.
\end{equation}
The data matrix is approximated by an almost orthogonal sparse part that codes the harmonic instruments $V_H = W_HW_H^T V$ and a non constrained NMF part that codes the percussive instruments $V_P = W_PH_P$. The main advantage of the SPNMF results is in the fact that the method has few parameters compared to constrained NMF~\cite{canadas2014percussive}and NMPCF~\cite{kim2011nonnegative} while still obtaining similar results~\cite{laroche2015structured}. 


%\subsection{Using regularized NMF for the harmonic part}



\subsection{Optimization Algorithm}

In order to achieve such a decomposition, we can use a measure of fit $D(x|y)$ between the data matrix $V$ and the estimated matrix $\tilde{V}$. $D(x|y)$ is a scalar cost function and in this article we compare in Section \ref{setup:divergence} the use of the Euclidean distance (Euc), the Kullback Leiber (KL) divergence and the Itakura Saito (IS) divergence which are three commonly used divergences in the NMF framework.



The SPNMF model gives the following cost function : 
\begin{equation}\label{InitCost}
\min_{W_H,W_P,H_P \geq 0} D(V|W_{H}W_{H}^{T}V + W_{P} H_{P})  
\end{equation}

A solution of this problem can be obtained by iterative multiplicative update rules following the same strategy as in~\cite{yuanOja2005,Lee01algorithmsfor} which consists in splitting the gradient with respect to (wrt) one variable (here $W_H$ for exemple) $\nabla_{W_H} D(V|\tilde{V})$ in its positive $[\nabla_{W_H} D(V|\tilde{V})]^{+}$ and negative parts $[\nabla_{W_H} D(V|\tilde{V})]^{-}$.
The multiplicative updates for SPNMF are then given by: 
$$W_{H} \leftarrow W_{H} \otimes \frac{ [\nabla_{W_H} D(V|\tilde{V})]^{-} }{[\nabla_{W_H} D(V|\tilde{V})]^{+}}, $$
where $\otimes$ is the Hadamard product or element-wise product. Details of the equations for the Euc distance, KL and IS divergences are given in the annex in Sections \ref{euclidisteq}, \ref{KLdisteq} and \ref{ISdisteq} respectively. 
The optimization of the SPNMF algorithm is done sequentially, as described in Algorithm \ref{AlgoMultipl}.

\begin{algorithm}[h]
 Input: $V \in \mathbb{R}_{+}^{m \times n} $
 Output: $W \in \mathbb{R}_{+}^{m \times k}$, $W_P \in \mathbb{R}_+^{m \times e}$ and $H_P \in \mathbb{R}_{+}^{e \times n}$
 Initialization\;
 \While{$i \leq$ number of iterations}{
	$W_{P} \leftarrow W_{P} \otimes \frac{ [\nabla_{W_P} D(V|\tilde{V})]^{-} }{[\nabla_{W_P} D(V|\tilde{V})]^{+}}$
	 \vspace{0.2cm}

	$H_{P} \leftarrow H_{P} \otimes \frac{ [\nabla_{H_P} D(V|\tilde{V})]^{-} }{[\nabla_{H_P} D(V|\tilde{V})]^{+}}$
	

	$W_{H} \leftarrow W_{H} \otimes \frac{ [\nabla_{W_H} D(V|\tilde{V})]^{-} }{[\nabla_{W_H} D(V|\tilde{V})]^{+}}$
	 \vspace{0.2cm}
	 	
	$i=i+1$ 
 }
 $ V_P = W_PH_P $ and
 $ V_H = W_HW_H^TV $ 
  
 \vspace{0.2cm}
 \caption{SPNMF algorithm with multiplicative update rules.}\label{AlgoMultipl}
\end{algorithm}


\subsection{SPNMF with fixed dictionary}\label{fixedict}  

A fully unsupervised SPNMF model does not allow for a satisfying harmonic/percussive source separation~\cite{laroche2015structured}. To alleviate this problem, we use here a fixed drum dictionary $W_p$ for the percussive part of the SPNMF which is created using the drum database ENST-Drums~\cite{gillet2006enst}. In this database three professional drum players specialized in a specific type of music have been recorded. Each drummer uses a specific drum set from a small one (two toms and two cymbals) to a full rock drum kit (four toms and five cymbals). 

Two different ways can be used to build a dictionary. The first one is to perform a NMF decomposition on a large database~\cite{jaureguiberry2011adaptation}. The rank of factorization is an important learning variable that will be discussed in a later section. This method is effective at giving a dictionary specific to an instrument and the size of the dictionary is maintained low so it does not increase the computation time. However, the template of the dictionary does not represent a single element of the drum kit so it is not possible to perform direct drum transcription.

A second way to build the dictionary is to directly use the STFT of a drum signal~\cite{wudrum}. The ENST-Drum database contains audio files where elements of the drum kit are played independently and that we used to build another types of drum dictionary. This method allows having a specific dictionary, yet the matrix is redundant and very large which increases the computation time. Both learning methods are compared in Section \ref{setup:dictionary}. \\

The main advantage of the proposed method over the previously mentioned concurrent approaches is that it can take into account some of the percussive instruments that have a sparse spectrum. The bass drum and the toms have almost harmonic spectra with most energy in the low frequency range. These sounds are consequently hard to extract while enforcing frequency regularity of the percussive part like in~\cite{canadas2014percussive,ono2008separation}.

The SPNMF algorithm with the fixed dictionary matrix is described by Algorithm \ref{AlgoDictionary}.
 
\begin{algorithm}[h]
 Input: $V \in \mathbb{R}_{+}^{m \times n} $
 Output: $W \in \mathbb{R}_{+}^{m \times k}$, $W_{train} \in \mathbb{R}_+^{m \times e}$ and $H \in \mathbb{R}_{+}^{e \times n}$
 Initialization\;
 \While{$i \leq$ number of iterations}{
	$H_{P} \leftarrow H_{P} \otimes \frac{ [\nabla_{H_P} D(V|\tilde{V})]^{-} }{[\nabla_{H_P} D(V|\tilde{V})]^{+}}$
	

	$W_{H} \leftarrow W_{H} \otimes \frac{ [\nabla_{W_H} D(V|\tilde{V})]^{-} }{[\nabla_{W_H} D(V|\tilde{V})]^{+}}$
	 \vspace{0.2cm}
	 	
	$i=i+1$ 
 }
 $ V_P = W_{train}H_P $ and
 $ V_H = W_HW_H^TV $ 
  
\vspace{0.2cm}
 \caption{SPNMF with the drum dictionary matrix.}\label{AlgoDictionary}
\end{algorithm}


 
 
\subsection{Signal reconstruction}

The percussive signal $x_p(t)$ is synthesized using the magnitude percussive spectrogram $V_P = W_PH_P$. To reconstruct the phase of the percussive part, we use a generalized Wiener filter~\cite{liutkus:hal-01110028} to create a percussive mask as:
\begin{equation}
\mathcal{M}_P = \frac{V_P^2}{V_M^2 + V_P^2},
\end{equation} 
\noindent, To retrieve the percussive signal as 
\begin{equation}
x_p(t) = InverseSTFT(\mathcal{M}_P \otimes X),
\end{equation}
\noindent where $X$ is the complex spectrogram of the mixture.

Similarly for the harmonic part, we obtain:
\begin{equation}\label{percuwiener}
\mathcal{M}_H = \frac{V_H^2}{V_M^2 + V_P^2},
\end{equation}
and:
\begin{equation}
x_h(t) = InverseSTFT(\mathcal{M}_H \otimes X).
\end{equation}


